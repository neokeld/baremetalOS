\documentclass{beamer}
\usepackage[francais]{babel}
\usepackage[utf8]{inputenc}
\usepackage[T1]{fontenc}
\usepackage{multicol}

\usetheme{Berlin}
\defbeamertemplate*{headline}{}
{
  \begin{beamercolorbox}[ht=2.25ex,dp=3ex,center]{section in head/foot}
    \insertnavigation{\paperwidth}
  \end{beamercolorbox}
}
 
\defbeamertemplate*{footline}{infolines}
{l
  \leavevmode%
  \hbox{%
  \begin{beamercolorbox}[wd=.5\paperwidth,ht=2.25ex,dp=1ex,center]{title in head/foot}
    \usebeamerfont{title in head/foot}\insertshorttitle
  \end{beamercolorbox}%
  \begin{beamercolorbox}[wd=.5\paperwidth,ht=2.25ex,dp=1ex,right]{date in head/foot}
    \usebeamerfont{date in head/foot}%\insertshortdate{}\hspace*{2em}
    ENSEIRB-MATMECA\hspace*{4em}
    \insertframenumber{} / \inserttotalframenumber\hspace*{4em}
  \end{beamercolorbox}}
  \vskip0pt%
}
\title{Projet Nmap}
\author{Mathias Brulatout \and Arnaud Duforat \\ Louis Lévêque \and Kamal Mallouky}
\institute{Responsables Pédagogiques : \\ Mathieu Blanc \& Pierre Lalet}
\titlegraphic{\includegraphics[scale=0.16]{enseirb.png}}
\date{\today}
\makeatletter
\newenvironment{withoutheadline}{
  \setbeamertemplate{headline}[default]
  \setbeamertemplate{footline}[default]
  \def\beamer@entrycode{\vspace*{-\headheight}}
}{}
\makeatother
\begin{document}
\begin{withoutheadline}
  \begin{frame}
    \titlepage
  \end{frame}
\end{withoutheadline}
\addtocounter{framenumber}{-1}
%KAMAL
\begin{frame}
  \frametitle{Introduction}
  \begin{itemize} % copie/colle pour faire une liste avec 3 items.
  \item Nmap.
  \item Nmap Scripting Engine.
  \item Lua.
  \end{itemize}
\end{frame}

\AtBeginSection[]
{
  \begin{frame}
    \frametitle{Sommaire}
    \tableofcontents[currentsection]
  \end{frame}
}
\begin{frame}
  \frametitle{Sommaire}
  \tableofcontents
\end{frame}


\section{Script Nmap}
\subsection{}
\begin{frame}
  \frametitle{Scripts Nmap}
  \begin{itemize}
    \item champs descriptifs (description/author/dependencies)
    
    \item section rules:
    \begin{itemize}
      \item prerule()
      \item hostrule(host)
      \item portrule(host,port)
      \item postrule()
    \end{itemize}
  \item méthode action()
  \end{itemize}

\end{frame}
\begin{frame}
  \frametitle{Clés SSH}
  
    Basé sur la crypto asymétrique
    \begin{itemize}
      \item clé publique
      \item clé privée
      \item passphrase
    \end{itemize}
    SSH garantie:
    \begin{itemize}
  \item la confidentialité
  \item l'intégrité
  \item l'authentification
    \end{itemize}
\end{frame}
%LOUIS

\section{Réalisation}
\subsection{ }
\begin{frame}
  \frametitle{R\'ealisation}
  \begin{itemize}
  \item Utilisation de la librairie libssh2
  \item SWIG (Simple Wrapper and Interface Generator)
  \item autoconf
  \end{itemize}
\end{frame}

\begin{withoutheadline} 
\begin{frame}
\frametitle{Autoconf}
\begin{figure}
\includegraphics[width=4.4cm]{acschema.png}
\caption{Processus d'exécution d'autoconf}
\end{figure}

\end{frame}
\end{withoutheadline} 
\begin{frame}
\frametitle{Ajout de fonctions C dans la machine Lua}
 Structure lua\_State \\
 Fonctions
 \begin{itemize}
 \item lua\_tostring
 \item lua\_pushnumber
 \end{itemize}

\end{frame}

\section{Difficultés rencontrées}
\subsection{ }

\begin{frame}
\frametitle{Difficultés}
\begin{itemize}
\item Tentative de création d'un wrapper de C pour Lua
\item AC\_CHECK\_LIB ne peut pas fonctionner avec les macros
\item Makefile.in nécessaire pour linkflags
\end{itemize}
\end{frame}

\begin{frame}
\frametitle{Difficultés}
\begin{itemize}
 \item lua\_pushcfunction n'a jamais fonctionné
 \item utilisation de lua\_register
 \item impossibilité d'ajouter à la machine virtuelle des macros
 \end{itemize}
\end{frame}
\begin{frame}
\frametitle{Difficultés}
\begin{itemize}
\item libssh2\_trace non fonctionnel
\item libssh2\_session\_last\_error
\item lecture des codes d'erreur dans le code source
\item nécessité de patcher la libssh2 pour debug
\end{itemize}
\end{frame}

\begin{frame}
  \frametitle{Des problèmes bénéfiques}
  \begin{itemize}
  \item Apprentissage du Lua
  \item Modifications sur un code source peu documenté 
  \item Suppression de code inutile dans Nmap 
  \end{itemize}
\end{frame}

\begin{frame}
  \frametitle{Wireshark}
  \begin{figure}
    \hspace*{-1cm}
    \includegraphics[height=0.7\textheight]{wireshirk.png}
  \end{figure}
\end{frame}

\section{Big-IP}
\subsection{ }
\begin{frame}
  \frametitle{Equipements Big-IP}
  Utile pour une archi redondante
  \begin{itemize}
  \item Local/Global Traffic Manager
  \item VPN
  \item Proxy
  \item Firewall applicatif
  \end{itemize}
\end{frame}

\let\origaddtocontents=\addtocontents
\def\dontaddtocontents#1#2{}
\let\addtocontents=\dontaddtocontents
\section*{Conclusion}
\let\addtocontents=\origaddtocontents

\begin{frame}
  \frametitle{Conclusion}
  \begin{itemize}
  \item Solution non fonctionnelle
  \item Prise en main de Nmap et Lua
  \item \textsf{SSH Key Acceptance checker} a un réel intérêt
  \end{itemize}
\end{frame}

\end{document}
