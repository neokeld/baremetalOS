\documentclass{beamer}
\usepackage[francais]{babel}
\usepackage[utf8]{inputenc}
\usepackage[T1]{fontenc}
\usepackage{multicol}

\usetheme{Berlin}
\defbeamertemplate*{headline}{}
{
  \begin{beamercolorbox}[ht=2.25ex,dp=3ex,center]{section in head/foot}
    \insertnavigation{\paperwidth}
  \end{beamercolorbox}
}
 
\defbeamertemplate*{footline}{infolines}
{l
  \leavevmode%
  \hbox{%
  \begin{beamercolorbox}[wd=.5\paperwidth,ht=2.25ex,dp=1ex,center]{title in head/foot}
    \usebeamerfont{title in head/foot}\insertshorttitle
  \end{beamercolorbox}%
  \begin{beamercolorbox}[wd=.5\paperwidth,ht=2.25ex,dp=1ex,right]{date in head/foot}
    \usebeamerfont{date in head/foot}%\insertshortdate{}\hspace*{2em}
    ENSEIRB-MATMECA\hspace*{4em}
    \insertframenumber{} / \inserttotalframenumber\hspace*{4em}
  \end{beamercolorbox}}
  \vskip0pt%
}

\title{Projet Olimex}
\author{Mathias Brulatout \and Arnaud Duforat \and Louis Lévêque \\ Kamal Mallouky \and Nicolas Sarlin}
\institute{Responsable Pédagogique : \\ Aymeric Vincent}
\titlegraphic{\includegraphics[scale=0.16]{enseirb.png}}
\date{\today}
\makeatletter
\newenvironment{withoutheadline}{
  \setbeamertemplate{headline}[default]
  \setbeamertemplate{footline}[default]
  \def\beamer@entrycode{\vspace*{-\headheight}}
}{}
\makeatother
\begin{document}
\begin{withoutheadline}
  \begin{frame}
    \titlepage
  \end{frame}
\end{withoutheadline}
\addtocounter{framenumber}{-1}



\begin{frame}
  \frametitle{Introduction}
\end{frame}

\AtBeginSection[]
{
  \begin{frame}
    \frametitle{Sommaire}
    \tableofcontents[currentsection]
  \end{frame}
}
\begin{frame}
  \frametitle{Sommaire}
  \tableofcontents
\end{frame}


\section{Existant}
\subsection{}
\begin{frame}
  \frametitle{OLinuXino A20}
\end{frame}


\begin{frame}
  \frametitle{Uboot}
\end{frame}


\section{UART}
\subsection{}
\begin{frame}
  \frametitle{Principe}
  Différents registres :
  \begin{itemize}
    \item Contrôle
    \item Status
    \item Data
  \end{itemize}
  \textsf{getc} \& \textsf{putc} \\
  Emulation d'un terminal
\end{frame}

\begin{frame}
\frametitle{Commandes possibles}
\begin{itemize}
  \item
  \item
  \item
\end{itemize}
\end{frame}

%\begin{frame}
%\frametitle{Autoconf}
%\begin{figure}
%\includegraphics[width=4.4cm]{acschema.png}
%\caption{Processus d'exécution d'autoconf}
%\end{figure}
%\end{frame}


\begin{frame}
\frametitle{ }
\end{frame}

\section{GPIO}
\subsection{ }

\begin{frame}
\frametitle{ }
\end{frame}

\section{Memory Management}
\subsection{ }
\begin{frame}
  \frametitle{}
\end{frame}

\let\origaddtocontents=\addtocontents
\def\dontaddtocontents#1#2{}
\let\addtocontents=\dontaddtocontents
\section*{Conclusion}
\let\addtocontents=\origaddtocontents

\begin{frame}
  \frametitle{Conclusion}
\end{frame}

\end{document}
