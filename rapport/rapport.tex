\documentclass[frenchb]{article}
\usepackage[T1]{fontenc}
\usepackage[utf8]{inputenc}
\usepackage[francais]{babel}
\usepackage{fancyhdr}
\usepackage{float}
\usepackage{lastpage}
\usepackage{fancyvrb} 
\usepackage{graphicx}
\usepackage{hyperref}

\usepackage{listings}


\usepackage{xcolor}

\definecolor{mygreen}{rgb}{0,0.6,0}
\definecolor{mygray}{rgb}{0.3,0.3,0.3}
\definecolor{mymauve}{rgb}{0.58,0,0.82}

\lstset{ %
  backgroundcolor=\color{white},   % cbackground color \usepackage{xcolor}
  basicstyle=\footnotesize,        % font size
  breakatwhitespace=false,         % autobreak only at whitespace
  breaklines=true,                 % sets automatic line breaking
  captionpos=b,                    % sets the caption-position to bottom
  commentstyle=\color{mygreen},    % comment style
  deletekeywords={...},            % delete keywords from the given language
  escapeinside={\%*}{*)},          % if you want to add LaTeX within your code
  extendedchars=true,              % lets you use non-ASCII characters; no UTF8
  frame=single,                    % adds a frame around the code
  keepspaces=true,                 % keeps spaces, to indent : columns=flexible
  keywordstyle=\color{blue},       % keyword style
  language=Octave,                 % the language of the code
  morekeywords={*,...},            % if you want to add more keywords to the set
  numbers=left,                    % where to put line number (none, left, right)
  numbersep=5pt,                   % how far the line-numbers are from the code
  numberstyle=\footnotesize\color{mygray}, % line number style
  rulecolor=\color{white},         % black for rules
  showspaces=false,                % show spaces, overrides 'showstringspaces'
  showstringspaces=false,          % underline spaces within strings only
  showtabs=false,                  % show tabs 
  stepnumber=1,                    % the step between two line-numbers.
    stringstyle=\color{mymauve},     % string literal style
  tabsize=2,                       % sets default tabsize to 2 spaces
  title=\lstname                   % filename \lstinputlisting; try caption
}

\lstdefinestyle{customc}{
  belowcaptionskip=1\baselineskip,
  breaklines=true,
  frame=L,
  xleftmargin=\parindent,
  language=C,
  showstringspaces=false,
  basicstyle=\normalsize\ttfamily,
  keywordstyle=\bfseries \color{green!40!black},
  commentstyle=\itshape \color{purple!40!black}, %itshape
  identifierstyle=\color{blue},
  stringstyle=\color{orange},
}
\lstset{escapechar=@,style=customc}


\pagestyle{fancy}
\lhead{Projet Olimex}
\cfoot{M.BRULATOUT - A.DUFORAT - L.L\'EV\^EQUE - K.MALLOUKY - N.SARLIN}
\rfoot{\thepage}
\renewcommand{\footrulewidth}{0.5pt}
\begin{document}
\topmargin        = 0.mm
\headheight       = 12.1638pt
\evensidemargin   = 10.mm
%\oddsidemargin    = 10.mm
%\textheight       = 215.mm
%\textwidth        = 145.mm
\begin{titlepage}
  \begin{center}
    \textsc{\LARGE Rapport du projet PR311 : Olimex}\\[0.5cm]
    \textsc{\Large Mathias BRULATOUT\\ Arnaud DUFORAT\\ Louis L\'EV\^EQUE\\ Kamal MALLOUKY\\ Nicolas SARLIN}\\[1.5cm]
    \textsc{\Large \today }\\[1.5cm]
    \begin{figure}[h!]
      \center
      \includegraphics[width=8cm]{enseirb.png}
    \end{figure}
    \vspace{1cm}
    \textsc{\Large Responsable Pédagogique : Aymeric Vincent}
  \end{center}  
\end{titlepage}
%\tableofcontents
\section*{Introduction}
Dans le cadre du projet PR311 Développement système, il a été choisi de développer sur un Olimex OLinuXino A20. Pour en savoir plus sur la programmation bas niveau, nous avons décidé d'implémenter un pseudo-OS,  plutôt que d'utiliser une plateforme UNIX pour développer quelque chose ayant un intérêt pédagogique bien plus limité.

Ce rapport explique les démarches que nous avons adoptées pour mener à bien ce projet, les choix que nous avons fait et les raisons pour lesquelles nous les avons faits.

Tout d'abord, l'A20 sera presenté, afin d'enchaîner sur le travail et les choix effectués.
\clearpage
\section{Présentation de la plateforme}

\subsection{OLinuXino A20}
L'Olinuxino est un nano-ordinateur, composé d'une carte mère distribuée en matériel libre basé sur un système sur puce. Il s'agit d'un système embarqué sur une seule puce ARM. Cette unique puce intègre un mircoprocesseur dual core, de la mémoire vive, une mémoire flash, un connecteur micro SD et différents connecteurs tels que des ports USB etc. La puce intégre aussi 3 connecteurs GPIO regroupant un ensemble de 160 pins controlables, et une connection série sur laquelle nos travaux ont été focalisés.

L'A20 était doté dès le départ d'un système Android stocké sur la mémoire flash de l'équipement. Cependant, faire un projet de développement système sur un système Android n'était pas d'un grand intérêt, ni même installer un système UNIX sur une carte SD et le faire tourner sur l'Olinuxino pour faire du développement. Notre curiosité nous a poussé plus loin et nous avions décidé de developper un pseudo OS en C qui permettra de communiquer sur le port série via l'utilisation d'un UART et de manipuler les GPIO de la carte.

Plutôt que de développer un OS baremetal et de passer plus de temps à faire fonctionner les divers périphériques plutôt que de les utiliser, nous avons choisi d'utiliser Uboot, chargeur de démarrage nous permettant d'envoyer un binaire pouvant utiliser les différents périphériques sur la carte et de l'exécuter.
 
\subsection{U-boot}

Uboot (Universal Bootloader) est un chargeur de démarrage écrit en C permettant d'amorcer un Système d'exploitation sur un équipement particulier et fonctionne sur plusieurs architectures. Il est en général exécuté lors de la mise sous tension de l'équipement dans le but d'initialiser les périphériques. Uboot démarre à partir d'une ROM, et peut utiliser un cache de données ou une puce du processeur comme mémoire vive.

Après avoir récupéré et compilé les sources les plus récentes de U-boot via le git approprié à notre plateforme, il nous a fallu répeter de nombreuses fois la même manipulation avant de se rendre compte que la version la plus récente de U-boot ne marchait pas. Il nous a donc fallu revenir à une version antérieure.
Pour mettre U-boot sur une carte Micro SD, deux partitions ont été crées : la première formatée en vfat qui est le format attendu par le bootloader et la deuxième en ext2 si on veut mettre des binaires dessus :

\begin{itemize}
  \item \textsf{mkfs.vfat /dev/sdb1}
  \item \textsf{mkfs.ext2 /dev/sdb2}
 \end{itemize}

Il ne restait plus qu'à écrire grâce à un dd sur /dev/sdb, U-boot avec un SPL (Secondary Program Loader), qui se appelé par le bootloader et qui chargera U-boot :

\begin{verbatim}
dd if=u-boot-sunxi/u-boot-sunxi-with-spl.bin of=/dev/sdb bs=1024 seek=8
\end{verbatim}

Après compilation du code sans l'inclusion de la moindre bibliothèque externe et avec un grand nombre d'options via la commande \textsf{arm-linux-gnueabihf-gcc}, il suffit d'envoyer le binaire généré sur l'Olimex relié par un port USB série en utilisant la commande ckermit du coté client et loadb du coté Olimex.
Après l'envoi du binaire de l'OS sur une adresse bien spécifique au niveau de l'Olimex, il ne reste plus qu'à se positionner sur l'adresse en question et exécuter le binaire.

\section{UART}
Une fois que nous avons réussi à compiler et exécuter notre propre programme sur la carte, il a fallu développer ce qui allait être la base de l'interface de notre système : la communication série.
Pour cela il nous faut utiliser un UART (Universal Asynchronous Receiver Transmitter), composant dédié à ce type de communication.
Puisque l'UART0 est doté d'un brochage séparé et que c'est celui utilisé par la console U-Boot, nous avons décidé de l'utiliser nous aussi.
En cherchant dans le manuel d'utilisation du processeur A20, nous avons rapidement trouvé les informations nécessaires à l'utilisation du premier UART intégré :
\begin{itemize}
\item L'adresse de base de l'UART à partir de laquelle sont placés tous ses registres.
\item La liste des registres de contrôle et de données de l'UART.
\end{itemize}
Après avoir réussi à envoyer un octet en assignant une valeur au registre DAT, nous avons écrit trois fonctions permettant d'initialiser la communication, d'envoyer et de recevoir un caractère.

Le fichier \textsf{uart.h} contient les différentes adresses de l'UART nécessairesà l'usage que l'on en fait, à savoir la simulation d'un terminal (lecture/écriture de caractères).

\begin{lstlisting}
 /*UART0 address*/
#define UART0 0x01C28000
/*UART0's FIFO Control Register Offset*/
#define FCR (1 << 1)
/*UART0's Line Status Register offset*/
#define LSR 5
/*Data Ready Bit : First bit in LSR*/
#define DR 0x1
/*TX Holding Register Empty Bit : Sixth bit in LSR*/
#define THRE (1 << 5)
\end{lstlisting}
\vspace*{-0.8cm}

Le registre FCR (FIFO Control Register) nous permet de désactiver la gestion de file, pour nous faciliter la tâche.
L'idée est ensuite de manipuler le registre LSR (Line Status Register) et son bit DR (Data Ready) pour récupérer un caractère lorsque ce bit est égal à 1. Cela nous permet de simuler une fonction \textsf{getc}.

 Le bit THRE (Tx Holding Register Empty) du même registre est à 1 lorsque le mode FIFO est désactivé et lorsqu'un caractère est prêt à être reçu. Il ne reste plus qu'à écrire le caractère reçu pour simuler un \textsf{putc}.

En se basant sur la fonction d'envoi, nous avons pour plus de commodité créer une fonction destinée à envoyer une chaîne de caractères.

L'étape suivante à été naturellement de réaliser l'interface console à proprement dite. Une fonction est chargée d'afficher le prompt, une autre qui doit être appelée à intervalles réguliers, s'occupe de recevoir les frappes clavier et de renvoyer les caractères reçus afin de fournir un retour à l'utilisateur. Cette même fonction stocke les caractères dans un buffer, et appelle la fonction d'exécution de commande si l'utilisateur a pressé la touche Entrée.

%\lstinputlisting[caption=Scheduler, style=customc]{../uart.c}
\section{GPIO}

\section{Memory Management}

\clearpage

\section*{Conclusion}

\noindent Développement le plus bas niveau qu'on ait jamais fait. Sans uboot, ça aurait été du baremetal\\
Utilisation du C sans bibliothèque externe\\
Utilisation d'UART / GPIO inconnus avant\\
Implémentation de malloc/free\\
on est des autiste : (int*) + 1 = int* + sizeof(int)\\


{\Large\textbf{OLDCONCLUSION}}
Bien que nous n'ayons pas réussi à faire fonctionner notre script, ce projet a été l'occasion de découvrir diverses technologies. En effet, les scripts Nmap étant développés en Lua nous avons été amenés a nous familiariser avec ce langage de scripts. De plus, puisque notre projet nécessitait d'ajouter à Nmap une dépendance sur la bibliothèque \textsf{libssh2}, il nous a fallu apprendre à utiliser un des outils du système de compilation GNU, \textsf{autoconf}. 

Ce qui a sans doute coûté sa réussite à ce projet sont les problèmes rencontrés tout au long de ce dernier. L'avancement s'est fait par petites étapes débouchant chaque fois sur un nouveau problème dont la résolution nous prenait parfois plusieurs jours. Cependant le sujet et sa problématique étaient plutôt intéressants.

\end{document}
